\documentclass[12pt]{article}
\usepackage{amsmath}
\usepackage{amsthm}
\usepackage{amsfonts}
\usepackage{graphicx}
\usepackage{epstopdf}
\usepackage{float}
\usepackage{fancyhdr}
\usepackage{subcaption}
\usepackage{pdfpages}
\theoremstyle{definition}

\pagestyle{fancy}
\fancyhf{}

\DeclareMathOperator*{\Max}{Max}
\DeclareMathOperator*{\Min}{Min}
\setlength{\parindent}{0cm}

\newtheorem{quest}{Question}
\newtheorem{iss}{Issue}
\newtheorem{riskmeasure}{Definition}
\newtheorem{theorem}{Theorem}
\newtheorem{example}{Example}
\newtheorem{prop}{Proposition}

\begin{document}



\section{Risk}

Risk is the human perception of uncertainty.  The mathematical model for uncertainty is probability theory. The economic model for human perception is utility theory.  Combining the two theories leads to the concept of the utility function of a random variable.  Given the standard assumptions on a utility function \(U'(x)>0,\,U''(x)\leq 0\), then the \emph{expected utility} \(\mathbb{E}[U(X)]\) is less than \(U(\mathbb{E}[X])\) by Jensen's inequality; implying that humans with such utility functions are \emph{risk averse}.  Practicable problems involving utility functions include investment and capital allocation.  However, while humans may reveal preferences through prices, the estimation of the utility function is essentially intractable: each human has unique, time dependent utility.  This has lead researchers to attempt to solve a different problem.  Starting with Markowitz in 1952 \cite{markowitz1952}, research has focused on minimizing \emph{risk} for a given level of expected \emph{gain}.  Improvements on Markowitz's framework typically focus on generalizing the mathematical representation of \emph{risk} to more diverse utility functions while retaining computational tractability.

\section{Capital}

Capital is the investment in an institution on which minimal obligations are placed on the re-payment of the investment.  Hence investment in a firm's equity is risky for the investor, but less risky for the institution.  Assuming risk-averse investors, the firm must pay more for equity than for debt.  A firm with a healthy amount of capital is not as risky to investors as a highly levered firm.  This leads to lower returns for the investor: due both to the equilibrium pressures of the market place raising demand for investment in the firm and from the accounting fact that there is less return on equity for a given income. Regardless of the actual level of capital, it behooves the firm to leverage its capital in such a way as to maximize its stockholder's risk adjusted return.  

\section{Capital and Risk}

A firm, regardless of industry, invests the stockholder's capital in risky ventures.  The firm acts as an informed intermediary in its market: putting investor's capital to use in ways that the investor alone may not have the skill set or financial leverage to realize.  The level of risk that the firm takes can destroy or create value for the stockholder. The firm can invest in risky ventures without adequate capital and returns commiserate with the risk.  This is true even for relatively risk-less investments: if a firm invests all its capital in treasury bonds it will return less than the market risk-free rate due to administrative overheard.  
\\
\\
The most fundamental decision a firm has to make is whether to invest in asset \(X_j\).  This decision is influenced by the following factors: 
\begin{enumerate}
\item The contractural or expected return of \(X_j\)
\item The incremental cost of \(X_j\)
\item The risk of \(X_j\)
\item Factors 1-3 for \(X_i,\,i\neq j\)
\end{enumerate}

Combining these factor into a rule or algorithm to make investment decisions is an efficient method to ensure consistency and value across the firm.  One such algorithm that has the benefit of being relatively linear is the following:
\begin{equation} \label{eq1} \text{Decision} = \begin{cases} \text{Invest in } X_j \text{ if } \mathbb{E}[X_j]-f-qk>0 \\
\text{Do not invest otherwise}
\end{cases}\end{equation}
Where \(f\) is the cost of funds, \(q\) is the cost of capital, and \(k\) is the capital allocated to \(X_j\).  Each variable in decision \ref{eq1} must be modeled, though it is common for \(q\) to be a ``target'' return.  The viability of \(q\) as a target return can be seen by rearranging equation \ref{eq1} as follows:
\[\frac{\mathbb{E}[X_j]-f}{k}>q\]
Thus the decision rule is essentially equivalent to requiring the adjusted return on \(X_j\) to be above a certain threshold.  \(\mathbb{E}[X_j]\) and \(k\) can both be modeled using the distribution of \(X_j\) if the firm uses a risk based approach to capital allocation.  

\section{Distribution of Profit}

Every firm has a profit distribution.  As capital is the buffer for negative events, measuring the probability of exceeding the buffer (the probability of insolvency) is of importance to bondholders.  This is reflected in the default frequency of firms by bond grade as measured by a ratings agency.  Often a firm's risk appetite statement will say something along the lines of ``we want to take risk to align ourselves with a BBB rating''.  Modeling risk given such a statement is not difficult conceptually.  Given a profit distribution, the firm merely has to pick the point on the distribution that represents a probability of insolvency consistent with the historical default frequency of a BBB rated company.  If currently the firm has excess capital, riskier assets may be acquired (with commiserate return); or vice versa.  However, it is unclear how targeting a bond rating is consistent with the goals of the firm's \emph{stockholders}.  Stockholders have limited downside and unlimited upside, while bondholders have limited upside and drastic downside.  Hence stockholders tend to prefer volatility while bondholders tend to prefer stability.   The payoff to stockholders at some time \(T\) is given by \((a-d)^+\) where \(a\) is the firm's assets, \(d\) is the firm's debt, and \((\cdot)^+=max(\cdot, 0)\).  Meanwhile, the payoff to bondholders at \(T\) is \(-(d-a)^+ +q\) where \(q\) is some percentage of \(d\) reflecting the present value of coupon payments.  Thus stockholders essentially are long a call option on the firm's assets while bondholders have short a put on the firm's assets.  As the value of both put and call options are increasing functions of volatility, stockholders are more likely to promote volatility while bondholders will prefer stability.  
\\
\\
Assuming that the risk appetite incorporates a statement on the probability of solvency, the risk measure for the firm should be the value at risk.  The allocation of this value at risk to the firm's assets should be done using another method: both for computational convenience and due to the lack of sub-additivity of the value at risk metric.  


\section{Risk Measures}

Consider a portfolio \(X\).  This portfolio is time dependent, possibly continuous, and can be represented by the time series \(X_{t_1},\, X_{t_2},...X_{t_N}\).  It is common to be concerned not with the value of \(X_{t_i}\), but the change in the value of \(X_{t_i}\).  This change tends to be defined differently depending on the asset in question.  Loan portfolios typically model \(X_{t_{i+1}}-X_{t_i}\) while security portfolios typically model \(\mathrm{ln}(X_{t_{i+1}}/X_{t_i})\).  For the purposes of this section, the model for the return time series is unimportant. For notational convenience, this return sequence will characterize \(X\): for example, the expected return will be written \(\mathbb{E}[X]\).  Unless otherwise noted, \(X=\sum_i X_i\) where \(X_i\) is an asset in the portfolio \(X\).

\begin{riskmeasure}
A \emph{Risk Measure} is a function \(\rho: X \to \mathbb{R}\).  
\end{riskmeasure}

To be consistent with common industry practice, in this document a higher \(\rho(X)\) will be associated with higher risk.  A risky portfolio will typically have \(\rho(X)>0\).  \(\rho(X)\) should be chosen in such a way that it is consistent with utility theory in the following sense: Fix \(\mathbb{E}[X_i]=\mathbb{E}[X_j]\).  Then
\[\rho(X_i) < \rho(X_j) \implies \mathbb{E}[U(X_i)] > \mathbb{E}[U(X_j)] \]
The hope is that \(\rho(X)\) satisfies the above condition for a broad set of \(U\).  Markowitz's definition of \(\rho(X)\) is only applicable for quadratic utility functions (or if \(X\) is Gaussian).  This shortcoming has led to the development of more sophisticated formulations of \(\rho(X)\).  An ``axiomatic'' formulation for \(\rho(X)\) is given in \cite{artzner1999} and is as follows:

\begin{riskmeasure}
A \emph{coherent} risk measure satisfies the following:
\begin{enumerate}
\item \(\mathbb{P}(X_j=a)=1 \implies \rho(X_i+X_j)=\rho(X_i)-a\) (risk free returns do not add risk) \label{coh1}
\item \(\rho(X_j+X_i) \leq \rho(X_i)+\rho(X_j)\) (diversification is beneficial) \label{coh2}
\item \(\mathbb{P}(X_j<X_i)=1 \implies \rho(X_j) > \rho(X_i)\) (worse outcomes increase risk) \label{coh3}
\item \(\rho(aX)=a\rho(X), \, a \in \mathbb{R} ^+ \) (doubling the portfolio doubles the risk) \label{coh4}

\end{enumerate}
\end{riskmeasure}

It is important to note that a coherent risk measure need not be consistent with utility theory \cite{tsanakas2003}.  \(\rho(X)\) can be written as \(f(\mathbf{1}),\, f:\mathbb{R}^n \to \mathbb{R}\), where \(f(\mathbf{u})=\rho\left(\sum_i u_i X_i\right)\).

\begin{theorem}  A coherent risk measure is convex in \(\mathbf{u}\). 
\end{theorem}

\begin{proof}

 From property \ref{coh2} and \ref{coh4},  \(f(a \mathbf{u}_1+b\mathbf{u}_2)\leq a f(\mathbf{u}_1)+bf(\mathbf{u}_2),\,a,\,b \in \mathbb{R}^+\)
Letting \(b=1-a\), convexity follows.

\end{proof}

Thus a coherent risk measure is computationally convenient from an optimization point of view.

\subsection{Use of Risk Measures}

A financial institution's goal is to maximize shareholder value subject to a risk constraint.  This goal manifests itself in a series of decisions that financial institutions must continuously make:

\begin{enumerate}
\item Should asset \(X_i\) be added to the bank's portfolio?
\item How much capital is required for a given set of assets?
\item What is the optimal allocation of assets?
\end{enumerate}

These questions are interconnected and can be informed by the judicial use of risk measures.  The following is a list of possible applications that answers each question:
\begin{enumerate}
\item The bank's new portfolio will have risk \(\rho(X+X_i)\leq \rho(X)+\rho(X_i)\).  Thus if the bank has sufficient appetite for \(\rho(X)+\rho(X_i)\) it is guaranteed to have sufficient appetite for \(\rho(X+X_i)\).
\item It is possible that on a risk adjusted basis some sets of assets are under or over performing.  Aggregating this risk can lead to informed decisions about the profitability and value added of varying sets of assets.  
\item The convexity of the risk measure can lead to computationally feasible optimization problems such as allocating sets of assets.  
\end{enumerate}

\subsection{Examples of Risk Measures}


\subsubsection{Standard Deviation Risk Measures}
\begin{riskmeasure}
\(\rho\) is a \emph{standard deviation}  based risk measure (denoted \(\rho_s)\)) if \(\rho(X, c)=-\mathbb{E}[X]+c\sqrt{\mathbb{E}[X^2]-\mathbb{E}[X]^2}=-\mu+ c \sigma,\,c\in \mathbb{R}^+\).
\end{riskmeasure}

\begin{prop}
\(\rho_s\) is not coherent.
\end{prop}

\begin{proof}
Consider two assets with the following possible outcomes:
\begin{center}
\begin{tabular}{ c| c c c }
  \(\Omega\) & \(\mathbb{P}(\omega)\) & \(X_1\) & \(X_2\) \\
  \hline
  \(\omega_1\)  &.5 & -3 & -2 \\
  \(\omega_2\)  & .5 & -1 & 4 \\
\end{tabular}
\end{center}

Clearly \(X_1 < X_2 \,\forall \, \omega \in \Omega\).  \(\rho_s(X_1)=2+c\), \(\rho_s(X_2)=-1+9c\).  Take \(c=1\).  Then \(\rho_s(X_1, 1)<\rho_s(X_2, 1)\), violating property 2.  

\end{proof}

Despite the lack of coherence, the standard deviation is a popular risk measure due to analytical and computational convenience and its coherence when \(X\) is Gaussian.  For a portfolio of correlated assets, \(\sigma_s\) of a portfolio is the following:
\[\sigma_s (X)=\sqrt{\sum_{i, j} \rho_{i, j} \sigma_i \sigma_j }\]
Where \(\rho_{i, j}\) is the correlation of \(X_i, X_j\).

\subsubsection{VaR}

\begin{riskmeasure}
\(\rho\) is the Value at Risk (denoted \(\rho_v\)) if  \[\rho_v(X, \alpha)=-\sup{ \{q \in \mathbb{R}:\mathbb{P}(X< q) \leq \alpha\} }\] For \(\alpha \in (0, 1)\).\end{riskmeasure}


\begin{example}

Consider the discrete random variable \(X_1\) with the following distribution.

\begin{center}
\begin{tabular}{c| c c c}
\(\Omega\) & \(\mathbb{P}(\omega)\) & \(X_1\) & \(\mathbb{P}(X_1 < X_1(\omega))\) \\
\hline
\(\omega_1\) & .5 & -2 & 0 \\
\(\omega_2\) & .5 & 4 & .5\\
\end{tabular}
\end{center}

Thus the Value at Risk for \(\alpha=.05\) is \(2\) since \(-2\) is the largest \(X_1\) such that \(\mathbb{P}(X_1 < X_1(\omega))\leq .05\).  If \(\mathbb{P}(\omega_1)=.04\) instead of \(.5\), the Value at Risk is \(-4\) since \(4\) is the largest \(X\) such that \(\mathbb{P}(X_1 < X_1 (\omega))\leq .05\).
\end{example}

\begin{example}
When the distribution of \(X\) is continuous, the Value at Risk is simply the quantile function of \(X\) at \(\alpha\).  Consider the Gaussian random variable \(X_1\). \[\rho_v(X_1)=-\sup{ \{q \in \mathbb{R}:\mathbb{P}(X_1< q) = \alpha \}}\]
\[ =  -\sup{ \{q \in \mathbb{R}:\mathbb{P}(\mu+\sigma Z < q) = \alpha \}}, \,Z \sim \mathcal{N}(0, 1)  \]
\[=-\sup{ \left\{q \in \mathbb{R}:\mathbb{P}\left(Z <\frac{q-\mu}{\sigma}\right) = \alpha \right\}}\]
\[=-\sup{ \left\{q \in \mathbb{R}:\frac{q-\mu}{\sigma} = \Phi^{-1}(\alpha) \right\}}\]

\[=-\mu-\sigma\Phi^{-1}(\alpha) \]
\[=-\mu+\sigma \Phi^{-1}(1-\alpha)\]
\end{example}

\begin{prop}
\(\rho_v\) is not coherent.
\end{prop}
\begin{proof}
Consider two independent assets with the following properties:
\[X_{1, 2}=\begin{cases}
-100 & \text{with probability .04}\\
5 & \text{with probability .96}

\end{cases} \]

Constructing the table for \(X_1+X_2\), 

\begin{center}
\begin{tabular}{c|  c c c}
\(\Omega\) & \(\mathbb{P}(\omega)\) & \(X_1+X_2\) &  \(\mathbb{P}(X_1+X_2 < X_1(\omega)+X_2(\omega))\)  \\
\hline
\(\omega_1\) & .0016 & -200 & 0 \\
\(\omega_2\) & .0384 & -95 & .0016\\
\(\omega_3\) & .0384 & -95 & .0016 \\
\(\omega_4\) & .9216 & 10 & .0768 \\
\end{tabular}
\end{center}


While \(\rho_v (X_1, .05)=\rho_v (X_2, .05)=-5\), \(\rho_v (X_1+X_2, .05)=95\) thus violating property 2.

\end{proof}

Value at Risk is not convex and makes optimization difficult and possibly dangerous. However, VaR has the benefit of being easy to backtest if \(\alpha\) is relatively large and the time horizon is relatively short (for example, market VaR is typically daily with \(\alpha=.05\)).  If the VaR is used as a measure of \emph{enterprise} risk, the time horizon is likely to be a year and the \(\alpha\) is likely to be very small.  Indeed, the VaR for the enterprise is usually chosen such that losses in excess of the VaR lead to insolvency.  While using VaR for this purpose makes it difficult to bakctest, it also lends its lack of coherence irrelevant: stockholders are unconcerned about losses that exceed insolvency.

\subsubsection{Expected Shortfall}

\begin{riskmeasure}\label{defShortfall}

\(\rho\) is the Expected Shortfall (denoted \(\rho_g\)) if \(\rho_g (X, \alpha)=\frac{1}{\alpha}\int_0 ^ \alpha \rho_v(X, u) du\)

\end{riskmeasure}

\begin{prop}
If \(X\) has a continuous density \(f(x)\), then \[\rho_g(X, \alpha)=-\frac{1}{\alpha}\int_{-\infty} ^ {-\rho_v(X, \alpha)} x f(x) dx\]

\end{prop}

\begin{proof}
Since \(f(x)\) is continuous, \(\rho_v(X, u)=-F^{-1}(u)\) where \(F(x)\) is the cumulative density function of \(X\).  Substituting into the integral in \ref{defShortfall}, 
\[\rho_g (X, \alpha)=-\frac{1}{\alpha}\int_0 ^ \alpha F^{-1}(u) du\]
\[=-\frac{1}{\alpha}\int_{F^{-1}(0)} ^ {F^{-1}(\alpha)} F^{-1}(F(x)) dF(x)\]
\[=-\frac{1}{\alpha}\int_{-\infty} ^ {-\rho_v(X, \alpha)} x f(x) dx\]

\end{proof}

\begin{example}
If \(X\) is Gaussian,
\[\rho_g (X, \alpha)=-\frac{1}{\alpha} \int_{-\infty} ^{ \Phi ^{-1}(\alpha)} ( \mu +\sigma y) \frac{e^{-y^2 /2}}{\sqrt{2\pi}} dy\]
\[=-\frac{1}{\alpha} \left(\mu\Phi(\Phi^{-1}(\alpha))  -\sigma \int_{-\infty} ^{ \Phi ^{-1}(\alpha)} d \left(\frac{e^{-y^2 /2}}{\sqrt{2\pi}} \right) \right)\]
\[=-\frac{1}{\alpha} \left(\mu\alpha  -\sigma \phi \left(\Phi^{-1}(\alpha)\right) \right)\]
\[=-\mu+\frac{\sigma \phi \left( \Phi^{-1}(\alpha)\right)}{\alpha}\]

\end{example}

\begin{prop}
\(\rho_g\) is coherent.\cite{tasche2001}
\end{prop}

\section{Allocating Risk}
\subsection{Methods for Allocation}
There are two methods of allocating or pricing risk at a firm.  The first is to use a suitable coherent risk measure and apply it to each asset in the firm (or, for convenience, some set of assets).  Once found, these metrics can be added to determine the firm's ``total'' risk (the bottom up approach).  Since the risk measure is sub-additive, this estimate will be guaranteed to be conservative.  However, there is no reason to believe that this estimate will be close to the amount of capital that the firm holds or desires to hold.  Building a capital structure from the ground up imposes significant model risk:  even small errors in modeling (either from misspecification, miscalculation, or simply estimation error) can become large when summed over an entire portfolio.  Additionally, even if the risk measure is correctly computed, the degree of conservatism is unknown.
\\
\\
The second method is to model the entire firm's return distribution and compute the level of capital required (the top down approach).  This value is then allocated to each asset in the portfolio.  While there may be significant model risk in such an approach as well, modeling the entire firm's distribution of returns tends to be somewhat parsimonious compared to a bottom-up approach.  In addition, the outputs of such a model can be checked against common sense and historical results.  Allocating risk to assets will then likely be at least rank order consistent.  This top down model is the approach assumed for the remainder of this section. 

\subsection{Risk Contributions}
Let \(\rho_i\) denote the risk allocated to each asset.  The following are two properties that \(\rho_i\) should have.
\begin{enumerate} 
\item \(\sum_i \rho_i(X_i)=\rho(X)\) \label{props1}
\item \(\frac{\mathbb{E}[X_i]}{\rho_i(X_i)} >\frac{\mathbb{E}[X]}{\rho(X)} \implies \frac{\mathbb{E}[X+h X_i]}{\rho(X+h X_i)}>\frac{\mathbb{E}[X]}{\rho(X)},\,\forall \,h\in[0, \epsilon]\) \label{props2}
\end{enumerate}
The last property indicates that adding an asset with risk adjusted return greater than the portfolio improves the portfolio's risk adjusted return.  These two properties completely determine \(\rho_i\) \cite{tasche2007}.

\begin{riskmeasure}
The \emph{Euler risk contribution} \(\rho_i\) is \[\frac{d\rho(X+hX_i)}{dh}\bigg|_{h=0}\]
\end{riskmeasure}

This definition satisfies properties (\ref{props1}) and (\ref{props2}) if \(\rho\) satisfies coherence property \ref{coh4}.

\begin{proof}
Recall that \(\rho(X)\) can be written as \(f(\mathbf{1}),\, f:\mathbb{R}^n \to \mathbb{R}\), where \(f(\mathbf{u})=\rho\left(\sum_i u_i X_i\right)\).  Notice the following:
\[\frac{d\rho(X+hX_i)}{dh}\bigg|_{h=0}=\frac{\partial f} {\partial u_i}\left(\mathbf{1}\right)\]

By coherence property \ref{coh4}, \(f\left(\lambda \mathbf{u}\right)=\lambda f\left(\mathbf{u}\right)\). Taking the derivative of the left hand side with respect to \(\lambda\), 
\[\frac{\partial f(\lambda \mathbf{u})}{\partial \lambda}=\sum_i f'(\lambda \mathbf{u})\frac{\partial \lambda \mathbf{u}}{\partial \lambda}=
\sum_i u_i \frac{\partial f\left(\lambda \mathbf{u}\right)} {\partial u_i}\]
Clearly the derivative of the right hand side is
\[f(\mathbf{u})\]
Setting these equal, 
\[\sum_i u_i \frac{\partial f\left(\lambda \mathbf{u}\right)} {\partial u_i}=f(\mathbf{u})\]
Letting \(\lambda=1\) and \(\mathbf{u}=\mathbf{1}\), property (\ref{props1}) follows.
\\
\\
To prove property (\ref{props2}), note that 
\[\frac{\mathbb{E}[X_i]}{\rho_i(X_i)} >\frac{\mathbb{E}[X]}{\rho(X)} \implies \frac{\mathbb{E}[X+hX_i]}{\rho(X+hX_i)}>\frac{\mathbb{E}[X]}{\rho(X)}\]
\[\equiv \frac{\mathbb{E}[X_i]}{\mathbb{E}[X]} >\frac{\rho_i(X_i)}{\rho(X)} \implies \frac{\mathbb{E}[X+hX_i]}{\mathbb{E}[X]}>\frac{\rho(X+hX_i)}{\rho(X)}\]
\[\equiv \frac{\mathbb{E}[X_i]}{\mathbb{E}[X]} >\frac{\rho_i(X_i)}{\rho(X)} \implies \frac{\mathbb{E}[hX_i]}{\mathbb{E}[X]}>\frac{\rho(X+hX_i)-\rho(X)}{\rho(X)}\]


\[\equiv \frac{h \rho_i(X_i)}{\rho(X)} \geq\frac{\rho(X+hX_i)-\rho(X)}{\rho(X)}\]

\[\equiv h\rho_i(X_i) \geq \rho(X+hX_i)-\rho(X)\]
\[\equiv h \frac{\partial f }{\partial u_i}(\mathbf{1}) \geq f(\mathbf{1}+\mathbf{h}_i)-f(\mathbf{1})\]
Where \(\mathbf{h}_i=[0,\,0,\,...,h,\,0,...]\).
\[\equiv h \frac{\partial f }{\partial u_i}(\mathbf{1}) \geq f(\mathbf{1})+h \frac{\partial f }{\partial u_i}(\mathbf{1})+\frac{1}{2}\frac{\partial^2 f }{\partial u_i ^2} (\mathbf{1}) h^2+O(h^3) -f(\mathbf{1})\]
\[\equiv h \frac{\partial f }{\partial u_i}(\mathbf{1}) \geq h \frac{\partial f }{\partial u_i}(\mathbf{1})+\frac{1}{2}\frac{\partial^2 f }{\partial u_i ^2} (\mathbf{1}) h^2+O(h^3) \]
Since \(f(\mathbf{u})\) is convex, the second derivative is positive.

\[\frac{d\rho(X+hX_i)}{dh}\bigg|_{h=0}=\lim_{\Delta h \to 0}\frac{\rho(X+(h+\Delta h)X_i)-\rho(X+hX_i)}{\Delta h} \]

\end{proof}


\subsection{Risk Contributions for the Standard Deviation}
Recall that \(\rho_s (X)=-\sum_i \mu_i + c\sqrt{\sum_{i, j} \rho_{i, j} \sigma_i \sigma_j }\).  Using the Euler risk contribution,
\begin{multline*} \rho_j(X_j)=-\frac{d}{dh}\left(\sum_{i} \mu_i + h\mu_j \right)  \bigg|_{h=0} \\+c\frac{d}{dh} \sqrt{\mathbb{V}\left(\sum_{i} X_i\right)+2h\sum_i \rho_{i, j} \sigma_i \sigma_j +h^2 \sigma_j ^2 }  \Bigg|_{h=0}\end{multline*}
\[=-\mu_j + c \frac{2 \sum_i \rho_{i, j} \sigma_i \sigma_j + 2 h \sigma_j ^2}{2 \sqrt{\mathbb{V}\left(\sum_{i}X_i\right)+2h\sum_i \rho_{i, j} \sigma_i \sigma_j +h^2 \sigma_j ^2 }}  \bigg|_{h=0}\]

\[=-\mu_j+c \frac{\text{Cov}(\sum_i X_i, X_j)}{\sqrt{\mathbb{V}(\sum_i X_i)}}\]

\subsection{Generality of the Standard Deviation Risk Contributions}

Note that \(c\) is arbitrary.  Therefore the standard deviation based risk measure can be set equal to some top down risk metric like the Value at Risk.  Given \(\rho_v (X)=k\), and setting \(\rho_s=\rho_v\), \(c=\frac{k+\mu}{\sigma}\).  Thus the standard deviation risk contributions can be used even with an alternate risk measure like the Value at Risk. 

\section{Modeling the Distribution of Profit}
For a large retail bank there are essentially four distinct drivers of risk: to wit, credit, market, liquidity risk (contagion), and operational risk.  Each of these four risks can be modeled separately but as of yet there is no parsimonious and harmonious model that encompasses all four.    A model that generates the distribution of returns taking into account these four risks would be sufficient for the computation of \emph{economic capital}.
\begin{riskmeasure}  
\emph{Economic Capital} is the capital required to finance the firm's assets sans regulatory constraints.
\end{riskmeasure}

Economic capital depends on the risk preferences of stockholders.  

\subsection{Credit Risk}

Credit risk is the risk that counter-parties will not fulfill their obligations.  There are two models that are particularly popular, with their descriptions given below.

\begin{tabular}{c| p{5cm}| p{5cm} }
  & Credit Risk \(^+\) & CreditMetrics \\
\hline
Strengths & Computationally tractable, simple implementation of latent variable, easy computation of risk contributions, possibility of integration with other risk models through the characteristic function & Widely used, Guassian latent variable is easy to understand, additional correlated risk factors trivial to add, integration of rating changes \\
\hline
Weaknesses & Tricky to add correlated latent variables, probabilities can become larger than one (though rare), very low predicted correlation between assets & Computationally lengthy (MC), requires market data to estimate, difficult to integrate with other models in an efficient manner \\

\end{tabular}

  


\begin{thebibliography}{9}

\bibitem{artzner1999}
P. Artzner, F. Delbaen, J.-M. Eber, and D. Heath.  \emph{Coherent measures of risk} Mathematical Finance, 9 (3):203-228, 1999.

\bibitem{markowitz1952}
Harry Markowitz, \emph{Portfolio Selection
}.
The Journal of Finance, Vol. 7, No. 1. (Mar., 1952), pp. 77-91.


\bibitem{tasche2001}
Dirk Tasche, 
\emph{Expected Shortfall: a natural coherent alternative
to Value at Risk}.
Economic Notes, 31(2):379–388, 2002a.

\bibitem{tsanakas2003}
Andreas Tsanakas, Evangelia Desli, \emph{Risk Measures and Theories of Choice}. British Actuarial Journal, 9(4), p.959-991.

\bibitem{tasche2007}
  Dirk Tasche, Carlo Acerbi,
  \emph{Euler Allocation: Theory and Practice}.
  Working paper, Zentrum Marhematic (SCA),
TU Munchen, Germany, August 2007.

\end{thebibliography}
\end{document}
